\documentclass[11pt]{article}
\usepackage[margin=0.75in]{geometry}
\usepackage{amsmath}
\usepackage{enumerate}
\usepackage{enumitem}
\usepackage{parskip}
\usepackage{hyperref}

\hypersetup{
    colorlinks=true,
    urlcolor=cyan,
}


\setlist[itemize,1]{label={}}
\setlist[itemize,2]{label={}}

\begin{document}
\pagestyle{empty}
% --------------------------------------------------------------------------------------------------

% Title
\begin{center} {\Large\bfseries Concord Rules of Go Quick Start Guide} \vskip 1.5em \end{center}
\vskip 2em%
\subsection*{If You Use Japanese/Korean Rules}
    \begin{itemize}
    \item \textbf{When passing, place a pass stone in your opponent's prisoner bowl.} This gives the opponent one point, but since games typically end with two passes, this does not change the result.
    Pass stones are used in the USA, France, and UK (AGA rules).
    \item \textbf{The first and last pass must be made by different players.} In rare cases, three passes at the end might be needed.
    Example: ..., B pass, W threat, B answer, W pass, B pass, \textbf{W pass}.
    This prevents pass fights.
    \item \textbf{Count every surrounded point as territory, even in seki.} Seki (mutual life positions) eyes are points.
    \item \textbf{When scoring is unclear, continue playing until all dead stones are captured.} This resolves disputes without changing the score, even in complex Life/Death, Seki \& Ko situations.
    \end{itemize}

\vskip 2em%
\subsection*{If You Use Chinese Rules}
    \begin{itemize}
    \item \textbf{White gets one extra point if White made the very first pass.} This makes the first pass worth half as much as a dame point (neutral point).
    This removes the one-point difference between area and territory rules that occurs in about half of even games. This is the same as Taiwan Go rules.
    \item \textbf{White gets one point for each handicap stone beyond the first.} For example, with 3 handicap stones, White gets 2 extra points.
    \end{itemize}

\vskip 2em%
\subsection*{If You Use AGA Rules}
    \begin{itemize}
    \item \textbf{Games end with two passes (not three).}
    In AGA rules, White always makes the last pass, which causes three passes in half of all games.
    In Concord Rules, the final pass must be made by a different player than whoever made the very first pass.
    These rules prevent pass fights.

    \item \textbf{When using area counting, White gets an extra point if White made the very first pass.}
    Black has no such privilege. This makes the first pass half as valuable as a dame point (neutral point).
    \end{itemize}


% --------------------------------------------------------------------------------------------------
\newpage

% Title
\begin{center} {\Large\bfseries Concord Rules of Go} \vskip 1.5em \end{center}

\small
\begin{itemize}
\item \textbf{Setup.} Go is played between Black and White using black and white stones, two prisoner containers and a board.
Before starting, players agree on:
    \begin{itemize}
    \item Player colors
    \item Board used
    \item Komi (point compensation for White)
    \item Handicap (if any)
    \item Counting method (Territory or Area)
    \end{itemize}

\item \textbf{Play.} Players take turns making plays.
Black goes first.
    \begin{itemize}
    \item \textbf{Board.} The board is a grid of intersecting lines.
    Intersections are initially empty.
    \item \textbf{A play.} A play is either a \textit{pass play} or a \textit{board play}.
    Each play uses one stone of the player's color.
    \item \textbf{Pass play.} The player puts the stone in the opponent's prisoner container.
    This is a ``pass stone''.
    \item \textbf{Liberty.} A stone has a liberty if there is a path along board lines to an empty intersection through other intersections with stones of the same color.
    \item \textbf{Board play.} The player puts the stone on an empty intersection and captures all opponent stones that have no liberties.
    Captured stones are moved from the board to the player's prisoner container.
    \item \textbf{Suicide is illegal.} A play is illegal if the stone just placed has no liberties after capturing opponent stones.
    \item \textbf{Repetition is illegal.} A board play is illegal if it recreates the same board position with the same player to move (situational superko).
    \item \textbf{Illegal play handling.} If a player makes an illegal play and the opponent challenges it before the next move, the illegal play must be taken back and the player must pass.
    The opponent may allow the player to choose any legal play instead of the required pass.
    \end{itemize}

\item \textbf{End of play.} Play ends after two passes in a row.
However, if the Last pass rule (see below) requires a third pass in a row, play continues to include it.
    \begin{itemize}
    \item \textbf{Last pass.} The first and last pass in the game must be made by different players.
    In rare cases, three passes at the end might be needed. For example: ..., B pass, W threat, B answer, W pass, B pass, W pass.
    \item \textbf{Dead Stone Removal by Agreement.} After play ends, players try to agree on which stones on the board are ``dead'' and should be removed.
    If they agree, each player removes the opponent's dead stones, adds them to their own prisoner container, and then scoring begins.
    If the players cannot agree, play resumes.
    \item \textbf{Dispute resolution.} If no agreement is reached, either player can say ``Let's play until all dead stones are captured'' (or similar).
    In that case, play resumes and players should try to capture all opponent stones they can with board plays.
    When play ends again (as defined in 'End of play'), all stones left on the board are considered alive.
    There is no Dead Stone Removal by Agreement step.
    Stones stay on the board as they are for scoring.
    \end{itemize}

\item \textbf{Scoring.} Players calculate their scores using the agreed counting method.
White adds the komi.
The player with more points wins.
    \begin{itemize}
    \item \textbf{Territory intersection.} An empty intersection belongs to a player's territory if there is no path from it to an opponent's stone along board lines through other empty intersections.
    \item \textbf{Territory counting.} A player's score = number of territory intersections + number of prisoners in their container.
    \item \textbf{Area counting.} A player's score = number of their stones on the board + number of territory intersections.
    White gets an extra point if White played the very first pass in the game.
    White gets an extra point for every handicap stone except the first (e.g., if Black has 3 handicap stones, White gets 2 points).
    \end{itemize}
\end{itemize}

\end{document}
