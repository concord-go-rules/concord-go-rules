\documentclass[11pt]{article}
\usepackage[margin=1in]{geometry}
\usepackage{amsmath}
\usepackage{enumerate}
\usepackage{enumitem}
\usepackage{parskip}
\usepackage{hyperref}

\hypersetup{
    colorlinks=true,
    urlcolor=cyan,
}


\setlist[itemize,1]{label=\textbullet}
\setlist[itemize,2]{label=\textbullet}

\begin{document}
% --------------------------------------------------------------------------------------------------

% Title
\begin{center} {\LARGE\bfseries Concord Go Rules} \vskip 3em \end{center}

\section*{Why Concord Go Rules?}

Concord Go Rules combine the best features of Japanese/Korean, Chinese, AGA and Taiwan rules.
They achieve these three important goals:

\begin{itemize}
    \item \textbf{Every point matters.} Small endgame mistakes can alter the final score by a single point. (Similar to Japanese/Korean rules, unlike AGA and Chinese rules.)
    \item \textbf{Both territory and area counting work.} Both methods always give the same score. (Similar to AGA rules, unlike Chinese and Japanese/Korean rules.)
    \item \textbf{Just Play It Out.} Games can always continue until all dead stones are captured without changing the score. (Similar to Chinese and AGA rules, unlike Japanese/Korean rules.)
\end{itemize}

\subsection*{Why This Matters}

Go organizations worldwide use different rule systems, each with their own strengths.
These include the American Go Association, Chinese Weiqi Association, European Go Federation, Kansai Ki-in, Korean Baduk Association, Nihon Ki-in, Taiwan Qixi Association, and others.
These differences confuse traveling players, complicate international events, and cause tournament disputes.
International players sometimes struggle with unfamiliar counting methods or have trouble agreeing on scoring complex seki/ko siutations.
For example, see \href{https://senseis.xmp.net/?RuleDisputesInvolvingGoSeigen}{rule disputes involving Go Seigen}.

Concord Go Rules show how these traditions can work together.

\subsection*{Built on Respect and Precision}

Key features:

\begin{itemize}
\item Resolves the key problem of simultanously supporting both: territory counting and Just Play It Out.
\item Mathematical proof that both counting methods give the same result.
\item Just two small changes: pass stones and a last pass rule
\end{itemize}

Concord Go Rules provide a path to common standards for the global Go community.

% --------------------------------------------------------------------------------------------------
\newpage
\section*{Concord Go Rules for Users of Popular Rulesets}
\vskip 2em%
\subsection*{Differences from Japanese/Korean Rules}
    \begin{itemize}
    \item \textbf{Every move uses one stone, even pass.} In Concord Go Rules, a pass requires placing a ``pass stone'' in the opponent's prisoner bowl.
    This gives the opponent one point, but since games typically end with two passes, this does not change the result.
    Pass stones are used in the USA, France and UK (AGA rules).
    \item \textbf{Just play it out.} When scoring is unclear, Concord Go Rules allow games to continue until all dead stones are captured without changing the score, even in complex Life/Death, Seki \& Ko situations.
    \item \textbf{Seki eyes are points.} Every surrounded point counts, even in seki.
    \item \textbf{Last pass.} The first and last pass in the game must be made by different players.
    Rare example where this matters: ..., B pass, W threat, B answer, W pass, B pass, \textbf{W pass}.
    \end{itemize}

\vskip 2em%
\subsection*{Differences from Chinese Rules}
    \begin{itemize}
    \item \textbf{First Pass is Valuable.} White (but not Black) gets one extra point if White made the game's very first pass. This makes the first pass worth half as much as a dame point.
    This removes the one-point difference between area and territory rules that occurs in about half of even games. This is similar to Taiwan rules.
    \item \textbf{Handicap Compensation.} White gets one point for each handicap stone Black receives beyond the first.
    \end{itemize}

\vskip 2em%
\subsection*{Differences from AGA Rules}
    \begin{itemize}
    \item \textbf{Game typically ends with two passes.} In AGA rules, White always makes the last pass, causing three passes in half of the games.
    In Concord, the final pass must be made by a different player than whoever made the very first pass.

    \item \textbf{Area Counting Adjustment.} When using area counting, White (but not Black) gets an extra point if White made the game's very first pass.
    This makes the first pass half as valuable as a dame point.
    \end{itemize}


% --------------------------------------------------------------------------------------------------
\newpage
\section*{The Rules}

\begin{itemize}
\item \textbf{Setup.} Go is played between Black and White using black and white stones, two prisoner containers and a board.
Before starting, players agree on:
    \begin{itemize}
    \item Player colors
    \item Board used
    \item Komi (point compensation for White)
    \item Handicap (if any)
    \item Counting method (Territory or Area)
    \end{itemize}

\item \textbf{Play.} Players take turns making plays.
Black goes first.
    \begin{itemize}
    \item \textbf{Board.} The board is a grid of intersecting lines.
    Intersections are initially empty.
    \item \textbf{A play.} A play is either a \textit{pass play} or a \textit{board play}.
    Each play uses one stone of the player's color.
    \item \textbf{Pass play.} The player puts the stone in the opponent's prisoner container.
    This is a ``pass stone''.
    \item \textbf{Liberty.} A stone has a liberty if there is a path along board lines to an empty intersection through other intersections with stones of the same color.
    \item \textbf{Board play.} The player puts the stone on an empty intersection and captures all opponent stones that have no liberties.
    Captured stones are moved from the board to the player's prisoner container.
    \item \textbf{Suicide is illegal.} A play is illegal if the stone just placed has no liberties after capturing opponent stones.
    \item \textbf{Repetition is illegal.} A board play is illegal if it recreates the same board position with the same player to move (situational superko).
    \item \textbf{Illegal play handling.} If a player makes an illegal play and the opponent challenges it before the next move, the illegal play must be taken back and the player must pass.
    The opponent may allow the player to choose any legal play instead of the required pass.
    \end{itemize}

\item \textbf{End of play.} Play ends after two passes in a row.
However, if the Last pass rule (see below) requires a third pass in a row, play continues to include it.
    \begin{itemize}
    \item \textbf{Last pass.} The first and last pass in the game must be made by different players.
    In rare cases, three passes at the end might be needed. For example: ..., B pass, W threat, B answer, W pass, B pass, W pass.
    \item \textbf{Dead Stone Removal by Agreement.} After play ends, players try to agree on which stones on the board are ``dead'' and should be removed.
    If they agree, each player removes the opponent's dead stones, adds them to their own prisoner container, and then scoring begins.
    If the players cannot agree, play resumes.
    \item \textbf{Dispute resolution.} If no agreement is reached, either player can say ``Let's play until all dead stones are captured'' (or similar).
    In that case, play resumes and players should try to capture all opponent stones they can with board plays.
    When play ends again (as defined in 'End of play'), all stones left on the board are considered alive.
    There is no Dead Stone Removal by Agreement step.
    Stones stay on the board as they are for scoring.
    \end{itemize}

\item \textbf{Scoring.} Players calculate their scores using the agreed counting method.
White adds the komi.
The player with more points wins.
    \begin{itemize}
    \item \textbf{Territory intersection.} An empty intersection belongs to a player's territory if there is no path from it to an opponent's stone along board lines through other empty intersections.
    \item \textbf{Territory counting.} A player's score = number of territory intersections + number of prisoners in their container.
    \item \textbf{Area counting.} A player's score = number of their stones on the board + number of territory intersections.
    White gets an extra point if White played the very first pass in the game.
    White gets an extra point for every handicap stone except the first (e.g., if Black has 3 handicap stones, White gets 2 points).
    \end{itemize}
\end{itemize}

% --------------------------------------------------------------------------------------------------
\newpage

\section*{Discussion and commentary}

\subsection*{Historical Context: Learning from AGA Rules}

The AGA rules were created with similar goals to unify counting methods.
At first, they added pass stones without the last pass rule, which caused pass fights.
To fix these pass fights, they had to add the ``White passes last'' rule, which sometimes requires 3 passes.
This stopped the pass fights but created only 2-point resolution scoring.

\href{https://go.org.nz/index.php/about-go/history-of-nz-rules-of-go}{New Zealand} also attempted a similar unification.

Concord Go Rules achieve high 1-point resolution by using a different last pass rule, which also follows the tradition of 2 passes ending the game.

\subsection*{Ikeda rules}
Ikeda \href{https://gobase.org/studying/rules/ikeda/}{wrote a book} looking for ideal Go rules for international tournaments.
\href{https://gobase.org/studying/rules/ikeda/?sec=e_rules}{Ikeda's preferred Territory rules I} almost always give the same result as his area rules III
- "provided no extra passes are made before the end of competitive play".
Concord Go Rules improve on these by making territory and area counting always give the same result, even when extra passes are made.

\subsection*{Taiwan rules}
\href{https://senseis.xmp.net/?TaiwanRules}{Taiwan rules} are Concord Go Rules using only area counting.

\subsection*{Two Button Go}
\href{https://senseis.xmp.net/?TwoButtonGo}{Two Button Go} rules with territory counting are the same as Concord Go Rules with territory counting.
Concord Go Rules improve on Two Button Go by not requiring additional physical buttons.


% --------------------------------------------------------------------------------------------------
\newpage
\section*{Equivalence of Area and Territory counting}

This section proves that the two counting methods give equivalent results under Concord Go Rules.

\textbf{Definitions of variables}

\begin{itemize}
\item $C_B$, $C_W$ - number of stones \textbf{Captured} by Black and White
\item $B_B$, $B_W$ - number of Black and White stones left on \textbf{Board}, not counting handicap
\item $H_1$ - number of handicap stones minus 1 (i.e., number of extra stones placed with the first Black move).
In an even game $H_1 = 0$
\item $P_B$, $P_W$ - number of \textbf{Passes} by Black and White
\item $E_B$, $E_W$ - number of \textbf{Empty} intersections surrounded by Black and White
\item $M_B$, $M_W$ - total number of \textbf{Moves} (board plays and passes) made by Black and White.
So $M_B = B_B + C_B + P_B$ and $M_W = B_W + C_W + P_W$
\item $M_\Delta = M_B - M_W$ - difference in number of moves, either 0 or 1. \\
It equals 1 if Black made the last move (pass).\\
It equals 1 if White made the very first pass.
\end{itemize}

\begin{align}
\text{Area score} &= (E_B + (B_B+H_1)) - (E_W + B_W + \text{komi} + M_\Delta + H_1) \\
&= (E_B + B_B) - (E_W + B_W + \text{komi} + M_\Delta) \\
&= (E_B + B_B - M_B) - (E_W + B_W - M_W) - \text{komi} \\
&= (E_B + B_B - B_B - C_B - P_B ) - (E_W + B_W - B_W - C_W - P_W) - \text{komi} \\
&= (E_B - C_B - P_B) - (E_W - C_W - P_W) - \text{komi} \\
&= \text{Territory score}
\end{align}

\end{document}
