\documentclass[11pt]{article}
\usepackage[margin=0.75in]{geometry}
\usepackage{amsmath}
\usepackage{enumerate}
\usepackage{enumitem}
\usepackage{parskip}
\usepackage{hyperref}
\usepackage{xcolor}
\usepackage{array}

\hypersetup{
    colorlinks=true,
    urlcolor=cyan,
}


\setlist[itemize,1]{label={}}
\setlist[itemize,2]{label={}}

\begin{document}
\pagestyle{empty}
% --------------------------------------------------------------------------------------------------

% Title
\begin{center} {\Large\bfseries Concord Rules of Go Commentary} \vskip 1.5em \end{center}

\section*{Why Concord Rules?}

Concord Rules combine the best features of Japanese/Korean, Chinese, AGA and Taiwan rules.
They achieve these three important goals:

\begin{itemize}
    \item \textbf{Every point counts.} The rules preserve single-point precision in final scores, capturing skill differences that 2-point Chinese and AGA scoring might miss. (Similar to Japanese/Korean rules, unlike AGA and Chinese rules.)
    \item \textbf{Both territory and area scoring work.} Both methods always give the same score. (Similar to AGA rules, unlike Chinese and Japanese/Korean rules.)
    \item \textbf{Play Settles Disputes.} When life and death is unclear, simply capture all the dead stones - the final score remains unchanged (no complex analysis needed). (Similar to Chinese and AGA rules, unlike Japanese/Korean rules.)
\end{itemize}

\subsection*{Why This Matters}

Go organizations worldwide use different rule systems, each with their own strengths.
These include the American Go Association, Chinese Weiqi Association, European Go Federation, Kansai Ki-in, Korean Baduk Association, Nihon Ki-in, Taiwan Qixi Association, and others.
These differences confuse traveling players, complicate international events, and cause tournament disputes.
International players sometimes struggle with unfamiliar scoring methods or have trouble agreeing on scoring complex seki/ko situations.
For example, see \href{https://senseis.xmp.net/?RuleDisputesInvolvingGoSeigen}{rule disputes involving Go Seigen}.

Concord Rules show how these traditions can work together.

\subsection*{Built on Respect and Precision}

Key features:

\begin{itemize}
\item Achieves the rare combination of territory scoring and Play Settles Disputes.
\item Mathematical proof that both scoring methods give the same result.
\item Just two small changes: pass stones and a last pass rule.
\end{itemize}

Concord Rules provide a path to common standards for the global Go community.

% --------------------------------------------------------------------------------------------------
\newpage
\section*{Discussion and commentary}

\subsection*{Historical Context: Learning from AGA Rules}

The AGA rules were created with similar goals to unify scoring methods.
At first, they added pass stones without the last pass rule, which caused pass fights.
To fix these pass fights, they had to add the ``White passes last'' rule, which sometimes requires 3 passes.
This stopped the pass fights but created only 2-point resolution scoring.

\href{https://go.org.nz/index.php/about-go/history-of-nz-rules-of-go}{New Zealand} also attempted a similar unification.

Concord Rules achieve precise 1-point resolution by using a different last pass rule, which also follows the tradition of 2 passes ending the game.

\subsection*{Ikeda rules}
Ikeda \href{https://gobase.org/studying/rules/ikeda/}{wrote a book} looking for ideal Go rules for international tournaments.
\href{https://gobase.org/studying/rules/ikeda/?sec=e_rules}{Ikeda's preferred Territory rules I} almost always give the same result as his area rules III
- "provided no extra passes are made before the end of competitive play".
Concord Rules improve on these by making territory and area scoring always give the same result, even when extra passes are made.

\subsection*{Taiwan rules}
\href{https://senseis.xmp.net/?TaiwanRules}{Taiwan rules} are Concord Rules using only area scoring.

\subsection*{Two Button Go}
\href{https://senseis.xmp.net/?TwoButtonGo}{Two Button Go} rules with territory scoring are the same as Concord Rules with territory scoring.
Concord Rules improve on Two Button Go by not requiring additional physical buttons.

The first button is reflected in the rule of additional white compensation for the first pass in area scoring.
The second button is reflected in the rule that first and last pass must be done by different players.

\subsection*{Alternative formulation}
With Concord area scoring, instead of asymmetric "White gets an extra point if White played the very first pass in the game.", we could have an integer komi and a rule that if the score is 0, the game is won by whoever passes first.
This is symmetric and closer to button go. The downside is that it is further from all classical rulesets, possibly hindering adoption.

\subsection*{Strategy consequences}
With Concord rules, all on-board intersections are now valuable, just like with Chinese and AGA rules.
Because of that, it is worth fighting for them and they have to be played as a part of the game.
Additionally, the first pass is also worth something, and while it is worth half of a dame, it might be worth more than the last ko for the player with more ko threats.
Here is an example of such a situation: https://online-go.com/demo/1499043

\subsection*{Pass fights}
There are no pass fights, because of the \textbf{Last pass} rule: "The first and last pass in the game must be made by different players."
It resolves the problem the same way the AGA rule "White must pass last" resolves pass fights.


% --------------------------------------------------------------------------------------------------
\newpage
\section*{Equivalence of area and territory scoring}

This mathematical proof demonstrates why both territory and area scoring methods always give identical results under Concord Rules, eliminating the traditional trade-off between scoring systems.

\textbf{Definitions of variables}

\begin{itemize}
\item $C_B$, $C_W$ - number of stones \textbf{Captured} by Black and White
\item $B_B$, $B_W$ - number of Black and White stones left on \textbf{Board}, not counting handicap
\item $H_1$ - number of handicap stones minus 1 (i.e., number of extra stones placed with the first Black move).
In an even game $H_1 = 0$. With 3 handicap stones, $H_1 = 2$
\item $P_B$, $P_W$ - number of \textbf{Passes} by Black and White
\item $E_B$, $E_W$ - number of \textbf{Empty} intersections surrounded by Black and White
\item $M_B$, $M_W$ - total number of \textbf{Moves} (board plays and passes) made by Black and White.
So $M_B = B_B + C_B + P_B$ and $M_W = B_W + C_W + P_W$
\item $M_\Delta = M_B - M_W$ - difference in number of moves, either 0 or 1. \\
It equals 1 if Black made the last move (pass).\\
It equals 1 if White made the very first pass.
\end{itemize}

With these definitions we can derive:
\begin{itemize}
    \item $(B_B+H_1)$ - number of Black stones on board.
    \item $(\text{komi} + M_\Delta + H_1)$ - additional (excluding these on board) points for White when using area scoring.
    \item $(E_B - C_B - P_B)$ - number of empty intersections surrounded by Black after filling in with the prisoners, including pass stones.
    \item $(E_W - C_W - P_W)$ - number of empty intersections surrounded by White after filling in with the prisoners, including pass stones.
\end{itemize}

Area score and territory score equality proof:
\begin{align}
\text{Area score} &= (E_B + (B_B+H_1)) - (E_W + B_W) - (\text{komi} + M_\Delta + H_1) \\
&= (E_B + B_B) - (E_W + B_W) - \text{komi} - (M_B - M_W)\\
&= (E_B + B_B - M_B) - (E_W + B_W - M_W) - \text{komi} \\
&= (E_B + B_B - B_B - C_B - P_B ) - (E_W + B_W - B_W - C_W - P_W) - \text{komi} \\
&= (E_B - C_B - P_B) - (E_W - C_W - P_W) - \text{komi} \\
&= \text{Territory score}
\end{align}

\end{document}
