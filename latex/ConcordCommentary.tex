\documentclass[11pt]{article}
\usepackage[margin=1in]{geometry}
\usepackage{amsmath}
\usepackage{enumerate}
\usepackage{enumitem}
\usepackage{parskip}
\usepackage{hyperref}
\usepackage{xcolor}
\usepackage{array}

\hypersetup{
    colorlinks=true,
    urlcolor=cyan,
}


\setlist[itemize,1]{label=\textbullet}
\setlist[itemize,2]{label=\textbullet}

\begin{document}
% --------------------------------------------------------------------------------------------------

% Title
\begin{center} {\LARGE\bfseries Concord Go Rules Commentary} \vskip 3em \end{center}

\section*{Why Concord Go Rules?}

Concord Go Rules combine the best features of Japanese/Korean, Chinese, AGA and Taiwan rules.
They achieve these three important goals:

\begin{itemize}
    \item \textbf{Every point matters.} Small endgame mistakes can alter the final score by a single point. (Similar to Japanese/Korean rules, unlike AGA and Chinese rules.)
    \item \textbf{Both territory and area counting work.} Both methods always give the same score. (Similar to AGA rules, unlike Chinese and Japanese/Korean rules.)
    \item \textbf{Resolution by Play.} Games can always continue until all dead stones are captured without changing the score (no complex analysis needed). (Similar to Chinese and AGA rules, unlike Japanese/Korean rules.)
\end{itemize}

\subsection*{Why This Matters}

Go organizations worldwide use different rule systems, each with their own strengths.
These include the American Go Association, Chinese Weiqi Association, European Go Federation, Kansai Ki-in, Korean Baduk Association, Nihon Ki-in, Taiwan Qixi Association, and others.
These differences confuse traveling players, complicate international events, and cause tournament disputes.
International players sometimes struggle with unfamiliar counting methods or have trouble agreeing on scoring complex seki/ko situations.
For example, see \href{https://senseis.xmp.net/?RuleDisputesInvolvingGoSeigen}{rule disputes involving Go Seigen}.

Concord Go Rules show how these traditions can work together.

\subsection*{Built on Respect and Precision}

Key features:

\begin{itemize}
\item Achieves the rare combination of territory counting and Resolution by Play.
\item Mathematical proof that both counting methods give the same result.
\item Just two small changes: pass stones and a last pass rule.
\end{itemize}

Concord Go Rules provide a path to common standards for the global Go community.

% --------------------------------------------------------------------------------------------------
\newpage
\section*{Discussion and commentary}

\subsection*{Historical Context: Learning from AGA Rules}

The AGA rules were created with similar goals to unify counting methods.
At first, they added pass stones without the last pass rule, which caused pass fights.
To fix these pass fights, they had to add the ``White passes last'' rule, which sometimes requires 3 passes.
This stopped the pass fights but created only 2-point resolution scoring.

\href{https://go.org.nz/index.php/about-go/history-of-nz-rules-of-go}{New Zealand} also attempted a similar unification.

Concord Go Rules achieve precise 1-point resolution by using a different last pass rule, which also follows the tradition of 2 passes ending the game.

\subsection*{Ikeda rules}
Ikeda \href{https://gobase.org/studying/rules/ikeda/}{wrote a book} looking for ideal Go rules for international tournaments.
\href{https://gobase.org/studying/rules/ikeda/?sec=e_rules}{Ikeda's preferred Territory rules I} almost always give the same result as his area rules III
- "provided no extra passes are made before the end of competitive play".
Concord Go Rules improve on these by making territory and area counting always give the same result, even when extra passes are made.

\subsection*{Taiwan rules}
\href{https://senseis.xmp.net/?TaiwanRules}{Taiwan rules} are Concord Go Rules using only area counting.

\subsection*{Two Button Go}
\href{https://senseis.xmp.net/?TwoButtonGo}{Two Button Go} rules with territory counting are the same as Concord Go Rules with territory counting.
Concord Go Rules improve on Two Button Go by not requiring additional physical buttons.


% --------------------------------------------------------------------------------------------------
\newpage
\section*{Equivalence of area and territory counting}

This mathematical proof demonstrates why both territory and area counting methods always give identical results under Concord Go Rules, eliminating the traditional trade-off between counting systems.

\textbf{Definitions of variables}

\begin{itemize}
\item $C_B$, $C_W$ - number of stones \textbf{Captured} by Black and White
\item $B_B$, $B_W$ - number of Black and White stones left on \textbf{Board}, not counting handicap
\item $H_1$ - number of handicap stones minus 1 (i.e., number of extra stones placed with the first Black move).
In an even game $H_1 = 0$. With 3 handicap stones, $H_1 = 2$
\item $P_B$, $P_W$ - number of \textbf{Passes} by Black and White
\item $E_B$, $E_W$ - number of \textbf{Empty} intersections surrounded by Black and White
\item $M_B$, $M_W$ - total number of \textbf{Moves} (board plays and passes) made by Black and White.
So $M_B = B_B + C_B + P_B$ and $M_W = B_W + C_W + P_W$
\item $M_\Delta = M_B - M_W$ - difference in number of moves, either 0 or 1. \\
It equals 1 if Black made the last move (pass).\\
It equals 1 if White made the very first pass.
\end{itemize}

\begin{align}
\text{Area score} &= (E_B + (B_B+H_1)) - (E_W + B_W + \text{komi} + M_\Delta + H_1) \\
&= (E_B + B_B) - (E_W + B_W + \text{komi} + M_\Delta) \\
&= (E_B + B_B - M_B) - (E_W + B_W - M_W) - \text{komi} \\
&= (E_B + B_B - B_B - C_B - P_B ) - (E_W + B_W - B_W - C_W - P_W) - \text{komi} \\
&= (E_B - C_B - P_B) - (E_W - C_W - P_W) - \text{komi} \\
&= \text{Territory score}
\end{align}

\end{document}